\subsection{GRBL software}

GRBL is an open source software that runs on an Arduino.
It accepts commands over the serial port and controls the stepper motors.

See documentation: \url{https://github.com/grbl/grbl}

\subsection{G200: Get buffer size}

We modified GRBL to accept a new G200 command.
When the controller receives a G200 command, it responds
with \verb|$G200=...| where \verb|...| is the amount of commands in the
buffer (that is, received commands that are not executed yet).
This way, our controller knows where in the execution the Arduino
currently is.

\subsection{M104: Set extruder and syringe temperatures}

We extended the \verb|M104| command to accept comma-separated tool--temperature tuples.  
The format is:

\begin{verbatim}
M104 T0:<temp>,T1:<temp>
\end{verbatim}

For example:
\begin{verbatim}
M104 T0:200,T1:180
\end{verbatim}

This sets the extruder (T0) to 200~°C and the syringe (T1) to 180~°C.  
Each tool index corresponds to a heater/thermistor pair in the firmware.  
If only one tuple is provided, only that tool's target temperature is updated.

\subsection{M105: Get extruder and syringe temperatures}

When the gcode sender sends \verb|M105|, the GRBL controller responds
with the current temperatures of all tools in the same tuple format:

\begin{verbatim}
$M105=T0:199.873,T1:180.122
\end{verbatim}

This makes it easy to monitor multiple heaters simultaneously.

\subsection{Temperature control}

The temperature can be set and monitored with the commands mentioned above.
In the source code of the controller, we added a file \verb|temperature.c| which
handles heating and temperature measurements. On every update cycle, the Arduino
reads the current temperature from each thermistor and decides whether the
corresponding heater should be switched on or off based on the target temperature.
This forms a simple bang-bang controller for each heater.
